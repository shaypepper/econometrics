\documentclass[]{article}
\usepackage{lmodern}
\usepackage{amssymb,amsmath}
\usepackage{ifxetex,ifluatex}
\usepackage{fixltx2e} % provides \textsubscript
\ifnum 0\ifxetex 1\fi\ifluatex 1\fi=0 % if pdftex
  \usepackage[T1]{fontenc}
  \usepackage[utf8]{inputenc}
\else % if luatex or xelatex
  \ifxetex
    \usepackage{mathspec}
  \else
    \usepackage{fontspec}
  \fi
  \defaultfontfeatures{Ligatures=TeX,Scale=MatchLowercase}
\fi
% use upquote if available, for straight quotes in verbatim environments
\IfFileExists{upquote.sty}{\usepackage{upquote}}{}
% use microtype if available
\IfFileExists{microtype.sty}{%
\usepackage{microtype}
\UseMicrotypeSet[protrusion]{basicmath} % disable protrusion for tt fonts
}{}
\usepackage[margin=1in]{geometry}
\usepackage{hyperref}
\hypersetup{unicode=true,
            pdftitle={Exam Questions},
            pdfborder={0 0 0},
            breaklinks=true}
\urlstyle{same}  % don't use monospace font for urls
\usepackage{color}
\usepackage{fancyvrb}
\newcommand{\VerbBar}{|}
\newcommand{\VERB}{\Verb[commandchars=\\\{\}]}
\DefineVerbatimEnvironment{Highlighting}{Verbatim}{commandchars=\\\{\}}
% Add ',fontsize=\small' for more characters per line
\usepackage{framed}
\definecolor{shadecolor}{RGB}{248,248,248}
\newenvironment{Shaded}{\begin{snugshade}}{\end{snugshade}}
\newcommand{\KeywordTok}[1]{\textcolor[rgb]{0.13,0.29,0.53}{\textbf{#1}}}
\newcommand{\DataTypeTok}[1]{\textcolor[rgb]{0.13,0.29,0.53}{#1}}
\newcommand{\DecValTok}[1]{\textcolor[rgb]{0.00,0.00,0.81}{#1}}
\newcommand{\BaseNTok}[1]{\textcolor[rgb]{0.00,0.00,0.81}{#1}}
\newcommand{\FloatTok}[1]{\textcolor[rgb]{0.00,0.00,0.81}{#1}}
\newcommand{\ConstantTok}[1]{\textcolor[rgb]{0.00,0.00,0.00}{#1}}
\newcommand{\CharTok}[1]{\textcolor[rgb]{0.31,0.60,0.02}{#1}}
\newcommand{\SpecialCharTok}[1]{\textcolor[rgb]{0.00,0.00,0.00}{#1}}
\newcommand{\StringTok}[1]{\textcolor[rgb]{0.31,0.60,0.02}{#1}}
\newcommand{\VerbatimStringTok}[1]{\textcolor[rgb]{0.31,0.60,0.02}{#1}}
\newcommand{\SpecialStringTok}[1]{\textcolor[rgb]{0.31,0.60,0.02}{#1}}
\newcommand{\ImportTok}[1]{#1}
\newcommand{\CommentTok}[1]{\textcolor[rgb]{0.56,0.35,0.01}{\textit{#1}}}
\newcommand{\DocumentationTok}[1]{\textcolor[rgb]{0.56,0.35,0.01}{\textbf{\textit{#1}}}}
\newcommand{\AnnotationTok}[1]{\textcolor[rgb]{0.56,0.35,0.01}{\textbf{\textit{#1}}}}
\newcommand{\CommentVarTok}[1]{\textcolor[rgb]{0.56,0.35,0.01}{\textbf{\textit{#1}}}}
\newcommand{\OtherTok}[1]{\textcolor[rgb]{0.56,0.35,0.01}{#1}}
\newcommand{\FunctionTok}[1]{\textcolor[rgb]{0.00,0.00,0.00}{#1}}
\newcommand{\VariableTok}[1]{\textcolor[rgb]{0.00,0.00,0.00}{#1}}
\newcommand{\ControlFlowTok}[1]{\textcolor[rgb]{0.13,0.29,0.53}{\textbf{#1}}}
\newcommand{\OperatorTok}[1]{\textcolor[rgb]{0.81,0.36,0.00}{\textbf{#1}}}
\newcommand{\BuiltInTok}[1]{#1}
\newcommand{\ExtensionTok}[1]{#1}
\newcommand{\PreprocessorTok}[1]{\textcolor[rgb]{0.56,0.35,0.01}{\textit{#1}}}
\newcommand{\AttributeTok}[1]{\textcolor[rgb]{0.77,0.63,0.00}{#1}}
\newcommand{\RegionMarkerTok}[1]{#1}
\newcommand{\InformationTok}[1]{\textcolor[rgb]{0.56,0.35,0.01}{\textbf{\textit{#1}}}}
\newcommand{\WarningTok}[1]{\textcolor[rgb]{0.56,0.35,0.01}{\textbf{\textit{#1}}}}
\newcommand{\AlertTok}[1]{\textcolor[rgb]{0.94,0.16,0.16}{#1}}
\newcommand{\ErrorTok}[1]{\textcolor[rgb]{0.64,0.00,0.00}{\textbf{#1}}}
\newcommand{\NormalTok}[1]{#1}
\usepackage{longtable,booktabs}
\usepackage{graphicx,grffile}
\makeatletter
\def\maxwidth{\ifdim\Gin@nat@width>\linewidth\linewidth\else\Gin@nat@width\fi}
\def\maxheight{\ifdim\Gin@nat@height>\textheight\textheight\else\Gin@nat@height\fi}
\makeatother
% Scale images if necessary, so that they will not overflow the page
% margins by default, and it is still possible to overwrite the defaults
% using explicit options in \includegraphics[width, height, ...]{}
\setkeys{Gin}{width=\maxwidth,height=\maxheight,keepaspectratio}
\IfFileExists{parskip.sty}{%
\usepackage{parskip}
}{% else
\setlength{\parindent}{0pt}
\setlength{\parskip}{6pt plus 2pt minus 1pt}
}
\setlength{\emergencystretch}{3em}  % prevent overfull lines
\providecommand{\tightlist}{%
  \setlength{\itemsep}{0pt}\setlength{\parskip}{0pt}}
\setcounter{secnumdepth}{0}
% Redefines (sub)paragraphs to behave more like sections
\ifx\paragraph\undefined\else
\let\oldparagraph\paragraph
\renewcommand{\paragraph}[1]{\oldparagraph{#1}\mbox{}}
\fi
\ifx\subparagraph\undefined\else
\let\oldsubparagraph\subparagraph
\renewcommand{\subparagraph}[1]{\oldsubparagraph{#1}\mbox{}}
\fi

%%% Use protect on footnotes to avoid problems with footnotes in titles
\let\rmarkdownfootnote\footnote%
\def\footnote{\protect\rmarkdownfootnote}

%%% Change title format to be more compact
\usepackage{titling}

% Create subtitle command for use in maketitle
\newcommand{\subtitle}[1]{
  \posttitle{
    \begin{center}\large#1\end{center}
    }
}

\setlength{\droptitle}{-2em}

  \title{Exam Questions}
    \pretitle{\vspace{\droptitle}\centering\huge}
  \posttitle{\par}
    \author{}
    \preauthor{}\postauthor{}
    \date{}
    \predate{}\postdate{}
  

\begin{document}
\maketitle

\subsection{1. Using the Normal and Student's T to find
p-values}\label{using-the-normal-and-students-t-to-find-p-values}

\subsubsection{Please answer the following. You may find it useful to
make a
sketch}\label{please-answer-the-following.-you-may-find-it-useful-to-make-a-sketch}

\paragraph{Example set 1 (30 points)}\label{example-set-1-30-points}

\begin{itemize}
\tightlist
\item
  For a Normal Distribution that has mean 11 and standard deviation 8.2
  , what isthe area to the right of 20.84 ?
\item
  For a Normal Distribution that has mean -5 and standard deviation 4 ,
  what is the area to the left of -6.6 ?
\item
  For a Normal Distribution that has mean 12 and standard deviation 4.5
  , what is the area to the left of 3.45 ?
\item
  For a Normal Distribution that has mean -14 and standard deviation 7.8
  , what is the area to the right of -28.82 ?
\item
  For a Normal Distribution that has mean 8 and standard deviation 9 ,
  what is the area to the left of -3.7 ?
\item
  For a Normal Distribution that has mean -10 and standard deviation 6.8
  , what is the area in both tails farther from the mean than 4.96 ?
\item
  For a Normal Distribution that has mean -6 and standard deviation 4.2
  , what is the area in both tails farther from the mean than -12.72 ?
\item
  For a Normal Distribution that has mean 10 and standard deviation 6.4
  , what is the area in both tails farther from the mean than -2.8 ?
\item
  For a Normal Distribution that has mean -6 and standard deviation 3.5
  what values leave probability 0.118 in both tails?
\item
  For a Normal Distribution that has mean -4 and standard deviation 0.1
  what values leave probability 0.024 in both tails?
\item
  A regression coefficient is estimated to be equal to 4.32 with
  standard error 2.7; there are 12 degrees of freedom. What is the
  p-value (from the t-statistic) against the null hypothesis of zero?
\item
  A regression coefficient is estimated to be equal to -19.11 with
  standard error 9.1; there are 40 degrees of freedom. What is the
  p-value (from the t-statistic) against the null hypothesis of zero?
\item
  A regression coefficient is estimated to be equal to -21.16 with
  standard error 9.2; there are 29 degrees of freedom. What is the
  p-value (from the t-statistic) against the null hypothesis of zero?
\end{itemize}

\paragraph{Example set 2 (25 points)}\label{example-set-2-25-points}

\begin{itemize}
\tightlist
\item
  For a Normal Distribution that has mean -6 and standard deviation 7.1,
  what is the area to the right of 3.94?
\item
  For a Normal Distribution that has mean 3 and standard deviation 6.9,
  what is the area to the right of 18.18?
\item
  For a Normal Distribution that has mean 12 and standard deviation 9.1,
  what is the area to the right of 5.63?
\item
  For a Normal Distribution that has mean 10 and standard deviation 8.7,
  what is the area to the right of 14.35?
\item
  For a Normal Distribution that has mean -12 and standard deviation
  1.1, what is the area to the left of -13.1?
\item
  For a Normal Distribution that has mean -2 and standard deviation 6.6,
  what is the area to the left of -5.96?
\item
  For a Normal Distribution that has mean 12 and standard deviation 6.3,
  what is the area in both tails farther from the mean than 8.85?
\item
  For a Normal Distribution that has mean 4 and standard deviation 7.3,
  what is the area in both tails farther from the mean than 14.95?
\item
  For a Normal Distribution that has mean 12 and standard deviation 5.2,
  what is the area in both tails farther from the mean than 5.76?
\item
  For a Normal Distribution that has mean 1 and standard deviation 3.1,
  what is the area in both tails farther from the mean than 7.82?
\item
  For a Normal Distribution that has mean -8 and standard deviation 1.8
  what values leave probability 0.278 in both tails?
\item
  For a Normal Distribution that has mean 14 and standard deviation 5.9
  what values leave probability 0.222 in both tails?
\item
  A regression coefficient is estimated to be equal to -7.2 with
  standard error 4.8; there are 10 degrees of freedom. What is the
  p-value (from the t-statistic) against the null hypothesis of zero?
\item
  A regression coefficient is estimated to be equal to -5 with standard
  error 2.5; there are 17 degrees of freedom. What is the pvalue (from
  the t-statistic) against the null hypothesis of zero?
\item
  A regression coefficient is estimated to be equal to 5.04 with
  standard error 2.4; there are 21 degrees of freedom. What is the
  p-value (from the t-statistic) against the null hypothesis of zero?
\end{itemize}

\paragraph{Example set 3 (15 points)}\label{example-set-3-15-points}

This set has answers

\begin{itemize}
\tightlist
\item
  For a Normal Distribution with mean of 12 and standard deviation 0.3,
  what is area to the right of 12.33?
\item
  For a Normal Distribution with mean of -2 and standard deviation 7.5,
  what is area to the right of 4?
\item
  For a Normal Distribution with mean of -2 and standard deviation 8.6,
  what is area to the left of -7.16?
\item
  For a Normal Distribution with mean of -9 and standard deviation 0.8,
  what is area to the left of -8.04?
\item
  For a Normal Distribution with mean 3 and standard deviation 8.9, what
  is area in both tails farther from the mean than -1.45?
\item
  For a Normal Distribution with mean 6 and standard deviation 6.9, what
  is area in both tails farther from the mean than -8.49?
\item
  For a Normal Distribution with mean -4 and standard deviation 5.8,
  what values leave probability 0.256 in both tails?
\item
  A regression coefficient is estimated to be equal to -0.6 with
  standard error 1; there are 23 degrees of freedom. What is the pvalue
  (from the t-statistic) against the null hypothesis of zero?
\item
  A regression coefficient is estimated to be equal to 3.04 with
  standard error 1.9; there are 8 degrees of freedom. What is the pvalue
  (from the t-statistic) against the null hypothesis of zero?
\end{itemize}

\paragraph{Example set 4}\label{example-set-4}

\begin{itemize}
\tightlist
\item
  For a Normal Distribution with mean 2 and standard deviation 5.6, what
  is area to the right of 10.4?
\item
  For a Normal Distribution with mean -1 and standard deviation 4.9,
  what is area to the right of -7.86?
\item
  For a Normal Distribution with mean 10 and standard deviation 5.2,
  what is area to the left of 0.12?
\item
  For a Normal Distribution with mean 7 and standard deviation 4, what
  is area to the left of 9.8?
\item
  For a Normal Distribution with mean 11 and standard deviation 8.4,
  what is area in both tails farther from the mean than 19.4?
\item
  For a Normal Distribution with mean -8 and standard deviation 0.4,
  what is area in both tails farther from the mean than -7.52?
\item
  For a Normal Distribution with mean 11 and standard deviation 6.3,
  what is area in both tails farther from the mean than 0.29?
\item
  For a Normal Distribution with mean 10 and standard deviation 1.5,
  what is area in both tails farther from the mean than 12.7?
\item
  For a Normal Distribution with mean -1 and standard deviation 1.2,
  what is area in both tails farther from the mean than 1.52?
\item
  For a Normal Distribution with mean 9 and standard deviation 4.9, what
  values leave probability 0.08 in both tails?
\item
  For a Normal Distribution with mean 3 and standard deviation 9.6, what
  values leave probability 0.311 in both tails?
\item
  A regression coefficient is estimated to be equal to 8.25 with
  standard error 7.5; there are 35 degrees of freedom. What is the
  p-value (from the t-statistic) against the null hypothesis of zero?
\item
  A regression coefficient is estimated to be equal to 10.08 with
  standard error 5.6; there are 7 degrees of freedom. What is the
  p-value (from the t-statistic) against the null hypothesis of zero?
\item
  A regression coefficient is estimated to be equal to -12.04 with
  standard error 8.6; there are 14 degrees of freedom. What is the
  p-value (from the t-statistic) against the null hypothesis of zero?
\end{itemize}

\paragraph{Example set 5}\label{example-set-5}

\begin{itemize}
\tightlist
\item
  If a variable has a Normal Distribution with mean 9 and standard
  deviation 9, what is area to the right of -8.3?
\item
  For a Normal Distribution with mean 5 and standard deviation 0.4, what
  is area to the left of 4.7?
\item
  For a Normal Distribution with mean 6 and standard deviation 0.3, what
  is area in both tails farther from the mean than 5.7?
\item
  For a Normal Distribution with mean -2 and standard deviation 3.8,
  what is area in both tails farther from the mean than 2.9?
\item
  For a Normal Distribution with mean 6 and standard deviation 7.5, what
  is area in both tails farther from the mean than -2.3?
\item
  For a Normal Distribution with mean 14 and standard deviation 3.4,
  what values leave probability 0.292 in both tails?
\item
  For a Normal Distribution with mean 8 and standard deviation 2.6, what
  values leave probability 0.253 in both tails?
\item
  For a Normal Distribution with mean -11 and standard deviation 2.6,
  what values leave probability 0.420 in both tails?
\item
  For a Normal Distribution with mean 2 and standard deviation 4.7, what
  values leave probability 0.007 in both tails?
\item
  For a Normal Distribution with mean -10 and standard deviation 7.9,
  what values leave probability 0.156 in both tails?
\end{itemize}

\paragraph{Example set 6}\label{example-set-6}

\begin{itemize}
\tightlist
\item
  For a Normal Distribution with mean 12 and standard deviation 8.9,
  what is area to the left of 4.9? A. 0.0888 B. 0.4237 C. 0.2125 D.
  0.7881
\item
  For a Normal Distribution with mean -2 and standard deviation 8.3,
  what is area to the left of -19.4? A. 0.9945 B. 0.0357 C.0.8531 D.
  0.0179
\item
  For a Normal Distribution with mean 3 and standard deviation 2.6, what
  is area to the right of 4.3? A. 0.1243 B. 0.3085 C.0.6171 D. 0.6915
\item
  For a Normal Distribution with mean 1 and standard deviation 2.3, what
  is area to the right of 0.8? A. 0.1587 B. 0.5346 C.0.3319 D. 0.5199
\item
  For a Normal Distribution with mean 5 and standard deviation 5.4, what
  is area in both tails farther from the mean than 10.4? A. 0.1587 B.
  0.6346 C. 0.8413 D. 0.3173
\item
  For a Normal Distribution with mean -7 and standard deviation 2.9,
  what is area in both tails farther from the mean than -5.3? A. 0.3821
  B. 0.1587 C. 0.7257 D. 0.5577
\item
  For a Normal Distribution with mean 11 and standard deviation 8.2,
  what is area in both tails farther from the mean than 1.2? A. 0.1587
  B. 0.2320 C. 0.8849 D. 0.7257
\item
  For a Normal Distribution with mean 8 and standard deviation 4.2, what
  values leave probability 0.046 in both tails? A. (-4.3807, 12.3807) B.
  (-0.0906, 3.9002) C. (-0.3807, 16.3807) D. (2.4201, 13.5799)
\item
  For a Normal Distribution with mean -6 and standard deviation 6.7,
  what values leave probability 0.347 in both tails? A. (-9.1504,
  -2.8496) B. (-1.8359, 0.0449) C. (-8.6360, -3.3640) D. (-12.3008,
  0.3008)
\item
  For a Normal Distribution with mean 5 and standard deviation 0.1, what
  values leave probability 0.301 in both tails? A. (4.9478, 5.0522) B.
  (4.8966, 5.1034) C. (4.9657, 5.0343) D. (-1.0343, 1.0343)
\end{itemize}

\paragraph{Example set 7}\label{example-set-7}

\begin{itemize}
\tightlist
\item
  For a Standard Normal Distribution, what is area to the left of 2.1?
  A. 0.5328 B. 0.0357 C. 0.3507 D. 0.9821
\item
  For a Standard Normal Distribution, what is area to the left of -0.6?
  A. 0.2743 B. 0.0547 C. 0.1368 D. 0.2002
\item
  For a Standard Normal Distribution, what is area to the left of 0.3?
  A. 0.3821 B. 0.9390 C. 0.6179 D. 0.7642
\item
  For a Standard Normal Distribution, what is area to the right of 0.9?
  A. 0.1841 B. 0.0875 C. 0.3684 D. 0.0552
\item
  For a Standard Normal Distribution, what is area to the right of -0.4?
  A. 0.5870 B. 0.3446 C. 0.6225 D. 0.6554
\item
  For a Standard Normal Distribution, what is area in both tails farther
  from the mean than 1.8? A. 0.9641 B. 0.1841 C. 0.1437 D. 0.0719
\item
  For a Standard Normal Distribution, what is area in both tails farther
  from the mean than 0.5? A. 0.0760 B. 0.5987 C. 0.6915 D. 0.6171
\item
  For a Standard Normal Distribution, what is area in both tails farther
  from the mean than -2.4? A. 0.9918 B. 0.0328 C. 0.0164 D. 0.0470
\item
  For a Standard Normal Distribution, what values leave probability
  0.324 in both tails? A. ±0.4565 B. ±1.2305 C. ±0.9863 D. ±0.3799
\item
  For a Standard Normal Distribution, what values leave probability
  0.390 in both tails? A. ±0.8596 B. ±0.8596 C. ±1.1948 D. ±0.4298
\item
  For a Standard Normal Distribution, what values leave probability
  0.218 in both tails? A. ±0.7790 B. ±0.1611 C. ±1.2319 D. ±1.2957
\end{itemize}

\paragraph{Example set 8}\label{example-set-8}

\begin{itemize}
\tightlist
\item
  For a Standard Normal Distribution, what is the area closer to the
  mean than 1.45?
\item
  For a Standard Normal Distribution, what is the area to the right of
  2?
\item
  For a Normal Distribution with mean 5 and standard deviation 7.6, what
  is area to the right of 14.1?
\item
  For a Normal Distribution with mean 1 and standard deviation 7.8, what
  is area in both tails farther from the mean than 11?
\item
  For a Normal Distribution with mean -5 and standard deviation 1.6,
  what is area in both tails farther from the mean than -2.6?
\item
  For a Normal Distribution with mean -1 and standard deviation 9.8,
  what values leave probability 0.157 in both tails?
\end{itemize}

\paragraph{Example set 9}\label{example-set-9}

\begin{itemize}
\tightlist
\item
  What is the probability, if the true distribution is a Standard
  Normal, of seeing a deviation from zero as large (in absolute value)
  as 1.9?
\item
  What is the probability, if the true distribution is a Standard
  Normal, of seeing a deviation from zero as large (in absolute value)
  as -1.5?
\item
  What is the probability, if the true distribution is a Standard
  Normal, of seeing a deviation as large (in absolute value) as 1.2?
\end{itemize}

\paragraph{Example set 10}\label{example-set-10}

\begin{itemize}
\tightlist
\item
  What is the probability, if the true distribution has mean -1 and
  standard deviation of 1.5, of seeing a deviation as large (in absolute
  value) as 2?
\item
  What is the probability, if the true distribution has mean 50 and
  standard deviation of 30, of seeing a deviation as large (in absolute
  value) as 95?
\item
  What is the probability, if the true distribution has mean 0.5 and
  standard deviation of 0.3, of seeing a deviation as large (in absolute
  value) as zero?
\end{itemize}

\paragraph{Example set 11}\label{example-set-11}

\begin{itemize}
\tightlist
\item
  What is the probability, if the true distribution is a Standard
  Normal, if seeing a value as large as 1.75?
\item
  What is the probability, if the true distribution is a Standard
  Normal, if seeing a value as large as 2?
\item
  If you observe a value of 1.3, what is the probability of observing
  such an extreme value, if the true distribution were Standard Normal ?
\item
  If you observe a value of 2.1, what is the probability of observing
  such an extreme value, if the true distribution were Standard Normal ?
\item
  What are the bounds within which 80\% of the probability mass of the
  Standard Normal lies?
\item
  What are the bounds within which 90\% of the probability mass of the
  Standard Normal lies?
\item
  What are the bounds within which 95\% of the probability mass of the
  Standard Normal lies?
\end{itemize}

\paragraph{Example set 12}\label{example-set-12}

\begin{itemize}
\tightlist
\item
  Find the area under the standard normal pdf between -1.75 and 0.
\item
  Find the area under the standard normal pdf between 0 and 1.75.
\item
  What is the probability of finding a value as large (in absolute
  value) as 1.75 or larger, if it truly has a standard normal
  distribution?
\item
  What values form a symmetric 90\% confidence interval for the standard
  normal (where symmetric means that the two tails have equal
  probability)? A 95\% confidence interval?
\end{itemize}

\paragraph{Example set 13}\label{example-set-13}

Consider a normal pdf with mean of 3 and standard deviation of 4. * Find
the area under the normal pdf between 3 and 7. * Find the area under the
normal pdf between 7 and 11. * What is the probability of finding a
value as far away from the mean as 7 if it truly has a normal
distribution?

\paragraph{Example set 14}\label{example-set-14}

\begin{itemize}
\tightlist
\item
  If a random variable is distributed normally with mean 2 and standard
  deviation of 3, what is the probability of finding a value as far from
  the mean as 6.5?
\item
  If a random variable is distributed normally with mean -2 and standard
  deviation of 4, what is the probability of finding a value as far from
  the mean as 0?
\item
  If a random variable is distributed normally with mean 2 and standard
  deviation of 3, what values form a symmetric 90\% confidence interval?
\item
  If a random variable is distributed normally with mean 2 and standard
  deviation of 2, what is a symmetric 95\% confidence interval? What is
  a symmetric 99\% confidence interval?
\end{itemize}

\paragraph{Example set 15}\label{example-set-15}

\begin{itemize}
\tightlist
\item
  What is the probability that we could observe a value as far or
  farther than 1.7?
\item
  What is the probability that we could observe a value nearer than 0.7?
\item
  What is the probability that we could observe a value as far or
  farther than 1.6?
\item
  What is the probability that we could observe a value nearer than 1.2?
\item
  What value would leave 15\% of the probability in the left tail?
\item
  What value would leave 10\% of the probability in the left tail?
\end{itemize}

\paragraph{Example set 16}\label{example-set-16}

A random variable is distributed with mean of 8 and standard deviation
of 4.

\begin{itemize}
\tightlist
\item
  What is the probability that we could observe a value lower than 6?
\item
  What is the probability that we could observe a value higher than 12?
\item
  What is the probability that we'd observe a value between 6.5 and 7.5?
\item
  What is the probability that we'd observe a value between 5.5 and 6.5?
\item
  What is the probability that the standardized value lies between 0.5
  and -0.5?
\end{itemize}

\paragraph{Example set 17}\label{example-set-17}

\begin{itemize}
\item
  What is the probability, if the true distribution is a Standard
  Normal, of seeing a deviation from zero as large (in absolute value)
  as 1.9?
\item
  What is the probability, if the true distribution is a Standard
  Normal, of seeing a deviation from zero as large (in absolute value)
  as -1.5?
\item
  What is the probability, if the true distribution is a Standard
  Normal, of seeing a deviation as large (in abs0lute value) as 1.2?*
\item
  What is the probability, if the true distribution has mean -1 and
  standard deviation of 1.5, of seeing a deviation as large (in absolute
  value) as 2?
\item
  What is the probability, if the true distribution has mean 50 and
  standard deviation of 30, of seeing a deviation as large (in absolute
  value) as 95?
\item
  What is the probability, if the true distribution has mean 0.5 and
  standard deviation of 0.3, of seeing a deviation as large (in absolute
  value) as zero?
\end{itemize}

\subsection{2. Confidence Intervals}\label{confidence-intervals}

\subsection{3. Statistics from given numbers (no datasets in R
required)}\label{statistics-from-given-numbers-no-datasets-in-r-required}

\paragraph{Problem 1. CI, HT}\label{problem-1.-ci-ht}

Consider an election with generic candidates X vs Y (which we could
interpret as chromosomes but not necessarily!).

\begin{enumerate}
\def\labelenumi{\alph{enumi}.}
\tightlist
\item
  Candidate X had the support of 54\% of the voters in the latest poll,
  which had 200 people. What is a 90\% confidence interval for the
  actual level of support?
\item
  If the polling organization were to add 100 more people to the sample
  size, how much would the 90\% confidence interval change?
\item
  Last month's poll had the candidate supported by 51\% -- both the
  previous poll and the current poll had 200 people. Is the difference
  in polling statistically significant?
\item
  Candidate X must win 2 particular states in order to win the election;
  the forecast says she has a 60\% chance of winning each state
  individually. Your friend, a wannabe statistician, explains that a 0.6
  chance of winning one state and a 0.6 chance of winning the other
  means only a 0.6*0.6= 0.36 chance of winning both - so the
  ``favorite'' is actually not the favorite! Explain why your friend is
  wrong.
\end{enumerate}

\paragraph{Problem 2.}\label{problem-2.}

(20 points) Suppose that a particular medical treatment already improves
patient outcomes by 20 (don't worry about the units for now) and it is
established that the standard deviation for the population is 8. There
is an improved treatment that is expected to deliver a further 10\%
improvement. a. If there were 10 patients in the trial, what would be
the t-statistic, p-value, and confidence interval -- assuming the new
treatment works as expected? Carefully explain the null hypothesis. b.
If there were 30 patients, what would be the t-stat, p-value, and
confidence interval (again assuming the treatment works as expected)? c.
If the company wants a p-value of 5\% or lower, how many patients should
they plan to have in the trial?

\paragraph{Problem 3.}\label{problem-3.}

\begin{enumerate}
\def\labelenumi{\arabic{enumi}.}
\setcounter{enumi}{8}
\tightlist
\item
  (20 points) Consider the following table of numbers of people (from
  CPS data) who make under or over \$15/hr in wage -- a level that some
  politicians want to set as the new minimum wage. (This is a particular
  subset, don't bother trying to replicate, the numbers given here
  should be sufficient.) Wage less than \$15/hr Wage greater than
  \$15/hr Native Immigrant Native Immigrant Educ HS or more 14235 3113
  33150 5296 no HS diploma 1062 1824 662 567
\end{enumerate}

\begin{enumerate}
\def\labelenumi{\alph{enumi}.}
\tightlist
\item
  Is the fraction of immigrants making less than \$15/hr different from
  the fraction making more than \$15/hr? In a statistical test of the
  difference, what is t-statistic, p-value, and confidence interval?
\item
  Is the fraction of people without a HS diploma, making less than
  \$15/hr, different from the fraction making more than \$15/hr? In a
  statistical test of the difference, what is t-statistic, p-value, and
  confidence interval?
\item
  In the population of people making less than \$15/hr, what fraction
  are immigrants without a HS diploma?
\item
  What is the conditional probability of finding an immigrant without a
  HS degree, given that the person is an immigrant and is making less
  than \$15/hr?
\item
  What is the conditional probability of finding an immigrant without a
  HS degree, given that the person is an immigrant and is making more
  than \$15/hr?
\end{enumerate}

\paragraph{Problem 4.}\label{problem-4.}

\begin{enumerate}
\def\labelenumi{\arabic{enumi}.}
\setcounter{enumi}{9}
\tightlist
\item
  (2o points) We consider the co-movement of unemployment and industrial
  production, comparing the change in the unemployment rate (UR) for
  months when industrial production (IP) rose against the change in the
  unemployment rate when industrial production fell, for months since
  1975. (Note that UR is measured in percentage points, so a change of
  .01 means UR went from 5\% to 5.01\%.) mean change in UR std. dev. of
  UR change N months when IP rose -.0498 .1513 325 months when IP fell
  .0815 .1836 162
\end{enumerate}

\begin{enumerate}
\def\labelenumi{\alph{enumi}.}
\tightlist
\item
  What is a t-statistic, p-value, and confidence interval for the null
  hypothesis that there is no difference?
\item
  Consider the same stats, but now looking only at the time since 2000,
  mean change in UR std. dev. of UR change N months when IP rose -.0349
  .1431 112 months when IP fell .0667 .1840 75
\item
  With the R-code below, can you find other relationships? Do these
  differences from above seem reasonable?
\end{enumerate}

\begin{Shaded}
\begin{Highlighting}[]
\KeywordTok{library}\NormalTok{(quantmod)}
\KeywordTok{getSymbols}\NormalTok{(}\KeywordTok{c}\NormalTok{(}\StringTok{'INDPRO'}\NormalTok{,}\StringTok{'UNRATE'}\NormalTok{),}\DataTypeTok{src=}\StringTok{'FRED'}\NormalTok{)}
\NormalTok{ip_}\DecValTok{1}\NormalTok{ <-}\StringTok{ }\NormalTok{INDPRO[}\StringTok{"1965::"}\NormalTok{]}
\NormalTok{ur_}\DecValTok{1}\NormalTok{ <-}\StringTok{ }\NormalTok{UNRATE[}\StringTok{"1965::"}\NormalTok{]}
\NormalTok{d_ip <-}\StringTok{ }\KeywordTok{na.trim}\NormalTok{(ip_}\DecValTok{1} \OperatorTok{-}\StringTok{ }\KeywordTok{lag}\NormalTok{(ip_}\DecValTok{1}\NormalTok{))}
\NormalTok{d_ur <-}\StringTok{ }\KeywordTok{na.trim}\NormalTok{(ur_}\DecValTok{1} \OperatorTok{-}\StringTok{ }\KeywordTok{lag}\NormalTok{(ur_}\DecValTok{1}\NormalTok{))}
\end{Highlighting}
\end{Shaded}

\paragraph{Problem 5. 20pts OLS}\label{problem-5.-20pts-ols}

Consider a simple regression where hours worked are regressed on a dummy
for whether the household is in a rural area and the omitted category is
that the person is in a more urban area. (This is a particular subsample
of the CEX but I'm not asking you to reestimate, you can figure the
answer from the information given here; there are 3595 degrees of
freedom). You get the following regression results.

\begin{longtable}[]{@{}lllll@{}}
\toprule
& Estimate & Std. Error & t value & Pr(\textgreater{}\tabularnewline
\midrule
\endhead
(Intercept) & \_\_\_\_\_ & 0.1422 & 311.677 & \_\_\_\_\_\tabularnewline
Rural\_DummyVariable & 0.7169 & \_\_\_\_\_ & 1.063 &
\_\_\_\_\_\tabularnewline
\bottomrule
\end{longtable}

\begin{enumerate}
\def\labelenumi{\alph{enumi}.}
\tightlist
\item
  Fill in the blanks in the table.
\item
  Suppose you instead looked at the difference in means, between hours
  worked by urban vs hours worked by rural households. How many hours
  per week do urban households work, on average? How many for rural? Is
  that a statistically significant difference?
\end{enumerate}

\paragraph{Problem 6. OLS !!!}\label{problem-6.-ols}

\begin{enumerate}
\def\labelenumi{\arabic{enumi}.}
\setcounter{enumi}{16}
\tightlist
\item
  (20 points) I will consider a simple question of the relation of
  employment to production -- relevant both for questions of ``jobless
  recovery'' and worker productivity. In the R dataset,
  ``macro\_data1.Rdata'', I give monthly data for the US on payroll
  (total nonfarm), the unemployment rate, and an index of industrial
  production for the period from February 1948 to August 2014. There is
  also a dummy variable for when the US was in a recession (as defined
  by NBER). The dataset has both the level of each of these (denoted
  lvl\_) and log difference (denoted ld\_), where 𝑙𝑑\_𝑧(𝑡) =
  𝑙𝑜𝑔(𝑙𝑣𝑙\_𝑧(𝑡)) − 𝑙𝑜𝑔(𝑙𝑣𝑙\_𝑧(𝑡 − 1)). You can use the command,
  load(``macro\_data1.RData''), to get the data in. I estimate the
  following regression for the period from 2000-date: 𝑑𝑃𝑎𝑦𝑟𝑜𝑙𝑙 𝑃𝑎𝑦𝑟𝑜𝑙𝑙 =
  .000739 + .0512 𝑑𝑃𝑟𝑜𝑑𝑢𝑐𝑡𝑖𝑜𝑛 𝑃𝑟𝑜𝑑𝑢𝑐𝑡𝑖𝑜𝑛 − .00270𝑅𝑒𝑐𝑒𝑠𝑠𝑖𝑜𝑛 The intercept
  coefficient has standard error of .00011, the slope coefficient on
  percent change in production has standard error or .0161, and the
  Recession dummy has standard error of .0003. The R-squared is 0.4943.
\end{enumerate}

\begin{enumerate}
\def\labelenumi{\alph{enumi}.}
\tightlist
\item
  What is the t-statistic for the slope coefficient? What is its
  p-value? (Carefully specify the null hypothesis.) What is a 95\%
  confidence interval for the slope coefficient?
\item
  Suppose that next month (not a recession month), the percent change in
  production is 0.004 -- what would the regression predict is the
  percent change in payroll?
\item
  How would you critique this regression? What might be improved?
\item
  Can you find some other interesting results from the data given?
  Explain.
\end{enumerate}

\paragraph{Problem 7. 10pts}\label{problem-7.-10pts}

One of the first notes in class emphasized ``know your data.'' You've
done a few homework assignments using Consumer Expenditure Survey data,
tell me about that data. How do they calculate expenditure on food away
from home?

\paragraph{Problem 8. OLS}\label{problem-8.-ols}

\begin{enumerate}
\def\labelenumi{\roman{enumi}.}
\tightlist
\item
  A regression coefficient is estimated to be equal to 1.902 with
  standard error 1.5; there are 26 degrees of freedom. What is the
  p-value (from the t-statistic) against the null hypothesis of zero?
\item
  A regression coefficient is estimated to be equal to 12.942 with
  standard error 9.6; there are 8 degrees of freedom. What is the
  p-value (from the t-statistic) against the null hypothesis of zero?
\item
  A regression coefficient is estimated to be equal to 3.647 with
  standard error 2.6; there are 15 degrees of freedom. What is the
  p-value (from the t-statistic) against the null hypothesis of zero?
\item
  A regression coefficient is estimated to be equal to -5.130 with
  standard error 3.5; there are 17 degrees of freedom. What is the
  p-value (from the t-statistic) against the null hypothesis of zero?
\item
  A regression coefficient has standard error 2.40; there are 14 degrees
  of freedom. The t-statistic is 2.5994. What is the coefficient?
\item
  A regression coefficient has standard error 3.40; there are 28 degrees
  of freedom. The t-statistic is -1.4877. What is the coefficient?
\item
  A regression coefficient has standard error 2.30; there are 12 degrees
  of freedom. The t-statistic is -1.0175. What is the coefficient?
\item
  A regression coefficient is estimated to be equal to 11.219; there are
  7 degrees of freedom. The t-statistic is 1.6259. What is the standard
  error?
\end{enumerate}

\paragraph{Problem 9. 20pts !!!}\label{problem-9.-20pts}

A recent research paper, looking at how much attractiveness and personal
grooming affects wages, used data from The National Longitudinal Study
of Adolescent Health in 2001-2. i. Are there gender differences? Among
the 6074 people (48.4\% female), 38.8\% of the males were rated as being
well groomed or very well groomed; 50.6\% of the females were rated that
way. Is this a statistically significant difference? ii. The study
considers interrelations between physical attractiveness and grooming.
People were ranked on a 4-point scale (where 1 is below average, 2 is
average, 3 is above average, and 4 is very much above average) for each
attribute. The full details are: Physically 4 Very Attractive 3
Attractive 2 Average 1 Less Attractive 4 Very well groomed 297 199 57 30
3 Well groomed 290 1169 607 54 2 Average grooming 75 788 2013 167 1 Less
than average grooming 1 25 164 138 Conditional on a person being ranked
physically 3 or 4 in attractiveness (above average), what is the chance
that they are above average (3 or 4) in grooming as well. Conditional on
being above average physically, what is the chance that they are average
or below average (1 or 2) in grooming? Are these statistically
significantly different? iii. The study also considers the
attractiveness of someone's personality (charisma), with the same
4-point scale. These data are: Personality 4 Very Attractive 3
Attractive 2 Average 1 Less Attractive 4 Very well groomed 326 171 60 26
3 Well groomed 416 1186 467 51 2 Average grooming 212 966 1729 136 1
Less than average grooming 11 49 184 84 Conditional on having an
above-average personality, what is the chance that someone has
above-average grooming? Conditional on having an above-average
personality, what is the chance that their grooming is at or below
average? Is there a statistically significant difference? iv. Comment on
the study. If overall attractiveness is a combination of these 3
factors, is there evidence that they are gross substitutes or
complements in production? PK Robins, JF Homer, MT French (2011).
``Beauty and the Labor Market: Accounting for the Additional Effects of
Personality and Grooming,'' Labour, 25(2), pp 228-251.

\paragraph{Problem 10. CI}\label{problem-10.-ci}

30 points. Cnfidence Intervals.

\begin{enumerate}
\def\labelenumi{\arabic{enumi}.}
\setcounter{enumi}{60}
\tightlist
\item
  You know that a random variable has a normal distribution with
  standard deviation of 16. After 10 draws, the average is -12.
\end{enumerate}

\begin{enumerate}
\def\labelenumi{\alph{enumi}.}
\tightlist
\item
  What is the standard error of the average estimate?
\item
  If the true mean were -11, what is the probability that we could
  observe a value between -10.5 and -11.5?
\end{enumerate}

\begin{enumerate}
\def\labelenumi{\arabic{enumi}.}
\setcounter{enumi}{61}
\tightlist
\item
  You know that a random variable has a normal distribution with
  standard deviation of 25. After 10 draws, the average is -10.
\end{enumerate}

\begin{enumerate}
\def\labelenumi{\alph{enumi}.}
\tightlist
\item
  What is the standard error of the average estimate?
\item
  If the true mean were -10, what is the probability that we could
  observe a value between -10.5 and -9.5
\end{enumerate}

\paragraph{Problem 11 !!!}\label{problem-11}

We use the most recent data to assess the relation between changes in
GDP and changes in the unemployment rate (so-called Okun's Law),
comparing the relation in the entire period since 1948 with the relation
in the period since 1990. Data are from FRED Stats. A regression has the
dependent variable as the quarterly change in the unemployment rate
(denotedUR). The independent variable is the quarterly percent growth
rate of nominal GDP (denoted \%Y). The estimated regression is  = + 
+ UR Y u   0 1\% . a. Using data for the entire period, 1948-2012, the
estimated equation is UR = 0.37 -- 0.22\%Y, where the standard error
of the intercept is 0.03, the standard error of the slope is 0.02, and
the R2 is 0.39. Is the slope coefficient statistically significant? What
is its t-statistic? P-value? b. Using data for the period 1990-2012, the
estimated equation is UR = 0.38 -- 0.30\%Y, where the standard error
of the intercept is 0.05, the standard error of the slope is 0.04, and
the R2 is 0.45. Is the slope coefficient statistically significant? What
is its t-statistic? P-value? c. Compare the two regressions. What are
the arguments in favor of using the whole sample versus only more recent
data? d. For the most recent data (first quarter of 2012), GDP growth
was 0.93 while UR was -0.4. What was the predicted value from each model
for that time? How would you interpret this?

\paragraph{Problem 12 !!!}\label{problem-12}

(15 points) I tracked down this reference from a sign on the bus, from
Tobacco Free NY. A survey of 1681 adolescents (age 11-14) in California
asked if they had tried smoking and how often they went to convenience,
liquor, or small grocery stores. The study finds that 452 kids rarely
went to these stores and 81 had tried cigarettes; 458 kids visited these
stores often (more than twice a week) and 133 had tried cigarettes. The
authors assert that visiting these stores exposed the kids to more
tobacco advertising. x. What is the difference in means? xi. What is the
standard error of the difference in means? xii. Is this difference
statistically significant? What is the p-value? Explain. xiii. The kids
were also asked if their grades were likely to be at the level of B or
below; 52 of the rare-frequency kids had belowaverage grades, while 63
of the high-frequency visitors had below-average grades. Is this
difference statistically significant? xiv. When asked about how often
they had seen tobacco advertising, low-frequency visitors reported a
mean of 3.1 (with standard error of 0.8) on a scale of 1-4 where 4 means
``often,''; high-frequency visitors reported a mean of 3.4 (with
standard error of 0.8). Is this difference statistically significant?
xv. Discuss the study; what else might you add? Hendrick, L, N C
Schleicher, E C Feighery, and S P Fortmann, (2010). ''Longitudinal Study
of Exposure to Retail Cigarette Advertising and Smoking Initiation,''
Pediatrics.

\paragraph{Problem 13. 15pts !!!}\label{problem-13.-15pts}

\begin{enumerate}
\def\labelenumi{\arabic{enumi}.}
\setcounter{enumi}{8}
\tightlist
\item
  (15 points) You might have missed this in the news about Alice Munro
  winning the Nobel, but there was a study done, showing that reading
  literature such as Munro and Chekov tended to make people score higher
  on psychological tests of Affective Theory of Mind. Consider the
  difference between two groups of people: either they read from a
  selection of literary fiction or they read non-fiction articles about
  non-human subjects (e.g.~potatoes). They were all given a test to
  determine how well they could identify emotion from a picture of a
  person's eyes. (I'm making up some of these numbers.) The Fiction
  group tests at 25.6 with standard deviation of 4.38; the Non-fiction
  group tests at 23.5 with standard deviation of 5.17. There were 41
  people in the first group and 45 in the second group.
\end{enumerate}

\begin{enumerate}
\def\labelenumi{\roman{enumi}.}
\setcounter{enumi}{15}
\tightlist
\item
  What is the difference in means?
\item
  What is the standard error of the difference in means?
\item
  Is this difference statistically significant? What is the p-value?
  Explain.
\item
  In another test, people read either literary fiction (that had won
  awards) or pop fiction (ie good sales but no awards). The lit fiction
  group scored 26.1 with standard deviation of 5.43 while the pop
  fiction group scored 23.7 with standard deviation of 5.08. Is this
  difference statistically significant?
\item
  Discuss the study, both in strengths and limitations. Kidd, D C, and E
  Castano, (2013). ``Reading Literary Fiction Improves Theory of Mind,''
  Science.
\end{enumerate}

\paragraph{Problem 14. !!!}\label{problem-14.}

. (20 points) You are comparing two groups: the first has X=0 and Y=1
and Y=3; the second has X=10 and Y=9 and Y=7. {[}So there are four data
points: (0,1), (0,3), (10,9), (10,7).{]} xxv. What is the difference in
means between the groups? xxvi. What is the standard error for the
difference in means? xxvii. Is the difference in means statistically
significant? xxviii. What is the slope of a regression line fitted to
the four points? xxix. What is the standard error of the slope? xxx. Is
the slope statistically significant? xxxi. If the Y-observations in the
second group were bigger, they might test as significant for the
difference in means. If the second group were (10,9+A) and (10,7+A),
what value(s) of A would make the difference significant? xxxii. Now if
the values of Y were changed to (10,9+B) and (10,7+B), what value(s) of
B would make the slope significant? xxxiii. What if, instead, the
X-values were changed by (10+C,9) and (10+C,7) -- what value(s) of C
would make the slope significant? xxxiv. Would changing X values change
the estimate for the difference in means? Explain.

\paragraph{Probelm 15. !!!}\label{probelm-15.}

(15 points) A (joking) study in the New England Journal of Medicine
linked a country's per capita consumption of chocolate with the number
of Nobel Prizes. It reported a regression coefficient but I got the data
and did my own analysis. Five countries with the highest consumption of
chocolate (UK, Belgium, Germany, Norway, Switzerland) had 19.02 Nobel
Prizes per 10m people (std dev 9.0); the next five countries (USA,
Finland, Denmark, Austria, Ireland) with lower chocolate consumption had
16.13 prizes (std dev 8.1). a. Construct a hypothesis test for whether
there is no statistically significant difference between the most
chocolate-consuming countries and the next group. What is the standard
error of the difference? What is the test statistic? What is the
p-value? b. Discuss the study. The countries that ate the most chocolate
consumed 9.28 kg/person (std dev 0.54) while the next group ate merely
7.34 kg/person (std dev 1.52). Should you have eaten a chocolate bar
before this exam? F H Messerli (2012). ``Chocolate Consumption,
Cognitive Function, and Nobel Laureates'' N Engl J Med 367: 1562-1564
October 18, 2012

\paragraph{Problem 16. !!!}\label{problem-16.}

(15 points) A (not joking) report from Morgan Stanley reported that the
bank's positions were 23\% safer because the bank chose to measure risk
with the standard deviation of stock returns from the past one year
rather than more years as it had done previously. Over the past one
year, the average daily returns on the S\&P500 (expressed at monthly
rate) were 1.7\% with a standard deviation of 3.6\% (252 observations).
For the four years before, the average return was -0.6\% with a standard
deviation of 6.5\% (1008 observations). Test the null hypothesis that
the returns for the past year are the same as the returns for the
previous four years. What is the standard error of the difference? What
is the test statistic? What is the p-value? Discuss. Why might Morgan
Stanley have chosen that particular data? (Note that riskiness is a cost
so reduces profits.) * The Excel file with this data is on Blackboard
although you do not need to use it. *

\paragraph{Problem 17 !!!}\label{problem-17}

\begin{enumerate}
\def\labelenumi{\arabic{enumi}.}
\setcounter{enumi}{18}
\tightlist
\item
  (15 points) A survey from eFinancialCareers found that, despite
  predictions from NY State that the Wall St bonus pool would drop by
  about 35\%, a full 48\% of the 911 respondents believed that their own
  bonuses would rise.
\end{enumerate}

\begin{enumerate}
\def\labelenumi{\alph{enumi}.}
\tightlist
\item
  Test the null hypothesis that the fraction of respondents expecting a
  bigger bonus is different from 35\%. What is the p-value?
\item
  Create a 95\% confidence interval for the fraction expecting a bigger
  bonus. What is the 90\% confidence interval? The 99\% interval?
\item
  Discuss. What other survey question might help explain this
  difference?
\end{enumerate}

\paragraph{Problem 18 !!!}\label{problem-18}

\begin{enumerate}
\def\labelenumi{\arabic{enumi}.}
\setcounter{enumi}{19}
\tightlist
\item
  (15 points) Dan Ariely and co-authors report a study that asks
  participants to solve complicated addition tasks but gives them an
  opportunity to cheat: they self-report how many problems they
  correctly solve. Every participant got a pair of fashion sunglasses
  but some were told that the sunglasses were counterfeit. Forty-two
  people were told they got counterfeit sunglasses and 30 of them
  cheated; 43 people were told that they got authentic sunglasses and 13
  of them cheated.
\end{enumerate}

\begin{enumerate}
\def\labelenumi{\alph{enumi}.}
\tightlist
\item
  Test the null hypothesis that there is no difference in the rates of
  cheating for people wearing counterfeit or authentic sunglasses. What
  is the p-value?
\item
  Create a 95\% confidence interval for the difference in cheating
  rates. What is the 90\% confidence interval? The 99\% interval?
\item
  Discuss. Are there other factors that could be at work? F Gino, M I
  Norton, D Ariely (2010). ``The Counterfeit Self: the Deceptive Costs
  of Faking It,'' Psychological Science 21:712.
\end{enumerate}

\paragraph{Problem 19 !!!}\label{problem-19}

\begin{enumerate}
\def\labelenumi{\arabic{enumi}.}
\setcounter{enumi}{20}
\tightlist
\item
  (15 points) An audit study emailed professors to ask for an
  appointment but the names of the `students' were randomly varied to be
  typically male or female; white, African-American, Hispanic, Chinese,
  or Indian. White men were 26\% more likely to get an appointment than
  minority women. Suppose you wanted to do a replication study for CUNY
  faculty. The original study emailed 6500 professors, you would like to
  study fewer.
\end{enumerate}

\begin{enumerate}
\def\labelenumi{\alph{enumi}.}
\tightlist
\item
  If the true difference in response to white male vs other names is
  26\%, what is the minimum sample size that could distinguish a
  statistically significant difference (at 95\% level)?
\item
  If the true difference were half as large, what is the minimum sample
  size to distinguish a difference?
\item
  What if the true difference were just 6\%? What is the minimum sample
  size necessary?
\item
  Discuss. Can you suggest additional tests? K L Mikman, M Akinola, D
  Chugh, 2012. ``Temporal Distance and Discrimination: An Audit Study in
  Academia,'' Psychological Science 23:7.
\end{enumerate}

\paragraph{Problem 20 !!!}\label{problem-20}

\begin{enumerate}
\def\labelenumi{\arabic{enumi}.}
\setcounter{enumi}{21}
\tightlist
\item
  (15 points) In recent news a study of adolescent girls compared those
  who had received a vaccination against HPV (a sexually transmitted
  virus that is linked to certain cancers) with those who had not
  received the vaccine. Some parents had been reluctant to get their
  children vaccinated because they believed this would encourage sexual
  activity. The study compared 493 who got the vaccine agains 905 who
  did not. Of the girls who got the vaccine, 61 got any of testing,
  diagnosis or counseling for pregnancy/sexuallytransmitted disease; of
  those who did not get the vaccine, 76 got testing, diagnosis, or
  counseling.
\end{enumerate}

\begin{enumerate}
\def\labelenumi{\alph{enumi}.}
\tightlist
\item
  Test the null hypothesis that there is no difference between outcomes
  for those who got the vaccine versus those who did not. What is the
  p-value for this difference?
\item
  Create a 95\% confidence interval for the difference between the
  groups. What is the 90\% confidence interval? The 99\% interval?
\item
  Discuss why this difference might be observed and how someone might
  critique the study. R A Bednarczyk, R Davis, K Ault, W Orenstein, S B
  Omer (2012). ``Sexual Activity-Related Outcomes After Human
  Papillomavirus Vaccination of 11- to 12-Year-Olds,'' Pediatrics.
\end{enumerate}

\paragraph{Problem 21 !!!}\label{problem-21}

\begin{enumerate}
\def\labelenumi{\arabic{enumi}.}
\setcounter{enumi}{25}
\tightlist
\item
  (15 points) In a medical study (reference below), people were randomly
  assigned to use either antibacterial products or regular soap. In
  total 592 people used antibacterial soap; 586 used regular soap. It
  was found that 33.1\% of people using antibacterial products got a
  cold; 32.3\% of people using regular soap got colds.
\end{enumerate}

\begin{enumerate}
\def\labelenumi{\alph{enumi}.}
\setcounter{enumi}{3}
\tightlist
\item
  Test the null hypothesis that there is no difference in the rates of
  sickness for people using regular or antibacterial soap. (What is the
  p-value?)
\item
  Create a 95\% confidence interval for the difference in sickness
  rates. What is the 90\% confidence interval? The 99\% interval?
\item
  Every other study has found similar results. Why do you think people
  would pay more for antibacterial soaps? E.L.Larson, S.X. Lin, C.
  Gomez-Pichardo, P. Della-Latta, (2004). ``Effect of Antibacterial Home
  Cleaning and Handwashing Products on Infectious Disease Symptoms: A
  Randomized Double-Blind Trial,'' Ann Intern Med, 140(5), 321-329.
\end{enumerate}

\paragraph{Problem 22 !!!}\label{problem-22}

\begin{enumerate}
\def\labelenumi{\arabic{enumi}.}
\setcounter{enumi}{26}
\tightlist
\item
  (15 points) A study of workers and managers asked both how much
  management listened to workers' suggestions (on a scale of 1-7 where
  ``1'' indicates that they paid great attention). Managers averaged a
  2.50 (standard deviation of 0.55); workers answered an average 2.08
  (standard deviation of 0.76) -- managers ignore their workers even
  more often than the employees realize. There were 137 workers and 14
  managers answering.
\end{enumerate}

\begin{enumerate}
\def\labelenumi{\alph{enumi}.}
\setcounter{enumi}{3}
\tightlist
\item
  Test the null hypothesis that there was no difference between workers
  and managers: how likely is it that there is actually no difference in
  average response? (What is the p-value?)
\item
  Create a 95\% confidence interval for the difference between workers
  and managers. What is the 90\% confidence interval? The 99\% interval?
\end{enumerate}

\paragraph{Problem 23 !!!}\label{problem-23}

\begin{enumerate}
\def\labelenumi{\arabic{enumi}.}
\setcounter{enumi}{27}
\tightlist
\item
  (15 points) A recent survey by Intel showed that 53\% of parents (561
  were surveyed) were uncomfortable talking with their children about
  math \& science. Previous surveys found that 57\% of parents talked
  with their kids about sex \& drugs.
\end{enumerate}

\begin{enumerate}
\def\labelenumi{\alph{enumi}.}
\tightlist
\item
  Test the null hypothesis that parents are as comfortable talking about
  math \& science as sex \& drugs; that the true value of parents
  uncomfortable with math and science is not different from 57\%. What
  is the p-value?
\item
  Create a 95\% confidence interval for the true fraction of parents who
  are uncomfortable with math \& science. What is the 90\% confidence
  interval? The 99\% interval?
\end{enumerate}

\paragraph{Problem 24 !!!}\label{problem-24}

\begin{enumerate}
\def\labelenumi{\arabic{enumi}.}
\setcounter{enumi}{28}
\tightlist
\item
  (15 points) The New York Times reported on educational companies that
  over-sell their products and gave the example of ``Cognitive Tutor''
  (CT) that helps math students. The CT students improved by 17.41
  (standard deviation of 5.82); the regular students improved by 15.28
  (standard deviation of 5.33). There were 153 students in the new
  program and 102 regular students.
\end{enumerate}

\begin{enumerate}
\def\labelenumi{\alph{enumi}.}
\setcounter{enumi}{3}
\tightlist
\item
  Test the null hypothesis that there is no difference between regular
  students and those in the CT group. What is the p-value for this
  difference?
\item
  Create a 95\% confidence interval for the difference between regular
  and CT students. What is the 90\% confidence interval? The 99\%
  interval?
\end{enumerate}

\paragraph{Probelm !!!}\label{probelm}

You are in charge of polling for a political campaign. You have
commissioned a poll of 300 likely voters. Since voters are divided into
three distinct geographical groups, the poll is subdivided into three
groups with 100 people each. The poll results are as follows: total A B
C number in favor of candidate 170 58 57 55 number total 300 100 100 100
std. dev. of poll 0.4956 0.4936 0.4951 0.4975 Note that the standard
deviation of the sample (not the standard error of the average) is
given. d. Calculate a t-statistic, p-value, and a confidence interval
for the main poll (with all of the people) and for each of the
subgroups. e. In simple language (less than 150 words), explain what the
poll means and how much confidence the campaign can put in the numbers.
f. Again in simple language (less than 150 words), answer the opposing
candidate's complaint, ``The biased media confidently says that I'll
lose even though they admit that they can't be sure about any of the
subgroups! That's neither fair nor accurate!''

\paragraph{Problem !!!}\label{problem}

Using data from the NHIS, we find the fraction of children who are
female, who are Hispanic, and who are African-American, for two separate
groups: those with and those without health insurance. Compute tests of
whether the differences in the means are significant; explain what the
tests tell us. (Note that the numbers in parentheses are the standard
deviations.) with health insurance without health insurance female
0.4905 (0.49994) N=7865 0.4811 (0.49990) N=950 Hispanic 0.2587 (0.43797)
N=7865 0.5411 (0.49857) N=950 African American 0.1785 (0.38297) N=7865
0.1516 (0.35880) N=950

\paragraph{Problem !!!}\label{problem-1}

For the ATUS dataset, use ``Analyze ~Descriptive Statistics ~Crosstabs''
to create a joint probability table showing the fractions of
males/females about the amount of time spent on the computer vs watching
TV (if either or both are above average). Find and interpret the joint
probabilities and marginal probabilities. Do this for age groups as
well.

\paragraph{Problem !!!}\label{problem-2}

A paper by Chiappori, Levitt, and Groseclose (2002) looked at the
strategies of penalty kickers and goalies in soccer. Because of the
speed of the play, the kicker and goalie must make their decisions
simultaneously (a Nash equilibrium in mixed strategies). For example, if
the goalie moves to the left when the kick also goes to the left, the
kick scores 63.2\% of the time; if the goalie goes left while the kick
goes right, then the kick scores 89.5\% of the time. In the sample there
were 117 occurrences when both players went to the left and 95 when the
goalie went left while the kick went right. What is the p-value for a
test that the probability of scoring is different? What advice, if any,
would you give to kickers, based on these results? Why or why not?

\paragraph{Problem !!!}\label{problem-3}

\begin{enumerate}
\def\labelenumi{\arabic{enumi}.}
\setcounter{enumi}{49}
\tightlist
\item
  A paper by Claudia Goldin and Cecelia Rouse (1997) discusses the
  fraction of men and women who are hired by major orchestras after
  auditions. Some orchestras had applicants perform from behind a screen
  (so that the gender of the applicant was unknown) while other
  orchestras did not use a screen and so were able to see the gender of
  the applicant. Their data show that, of 445 women who auditioned from
  behind a screen, a fraction 0.027 were ``hired''. Of the 599 women who
  auditioned without a screen, 0.017 were hired. Assume that these are
  Bernoulli random variables. Is there a statistically significant
  difference between the two samples? What is the p-value? Explain the
  possible significance of this study.
\end{enumerate}

\paragraph{Problem !!!}\label{problem-4}

\begin{enumerate}
\def\labelenumi{\arabic{enumi}.}
\setcounter{enumi}{50}
\tightlist
\item
  Another paper, by Kristin Butcher and Anne Piehl (1998), compared the
  rates of institutionalization (in jail, prison, or mental hospitals)
  among immigrants and natives. In 1990, 7.54\% of the institutionalized
  population (or 20,933 in the sample) were immigrants. The standard
  error of the fraction of institutionalized immigrants is 0.18. What is
  a 95\% confidence interval for the fraction of the entire population
  who are immigrants? If you know that 10.63\% of the general population
  at the time are immigrants, what conclusions can be made? Explain.
\end{enumerate}

\paragraph{Problem !!!}\label{problem-5}

You are consulting for a polling organization. They want to know how
many people they need to sample, when predicting the results of the
gubernatorial election. a. If there were 100 people polled, and the
candidates each had 50\% of the vote, what is the standard error of the
poll? b. If there were 200 people polled? c. If there were 400 people
polled? d. If one candidate were ahead with 60\% of the vote, what is
the standard error of the poll? e. They want the poll to be 95\%
accurate within plus or minus 3 percentage points. How many people do
they need to sample?

\subsection{4. Regression Analysis from given
data}\label{regression-analysis-from-given-data}

\paragraph{Problem 1}\label{problem-1-1}

To investigate an hypothesis proposed by a student, I got data, for 102
of the world's major countries, on the fraction of the population who
are religious as well as the income per capita and the enrollment rate
of boys and girls in primary school. The hypothesis to be investigated
is whether more religious societies tend to hold back women. I ran two
separate models: Model 1 uses girls enrollment rate as the dependent;
Model 2 uses the ratio of girls to boys enrollment rates as the
dependent. The results are below (standard errors in italics and
parentheses below each coefficient):

\begin{longtable}[]{@{}lllll@{}}
\toprule
& Model 1 & Model 2 & t-stat & p-value\tabularnewline
\midrule
\endhead
Intercept & 137 & 1.12 & &\tabularnewline
& (18) & (0.09) & &\tabularnewline
Religiosity & -0.585 & -0.0018 & &\tabularnewline
& (0.189) & (0.0009) & &\tabularnewline
GDP per capita & 0.00056 & 0.0000016 & &\tabularnewline
& (0.00015) & (0.0000007) & &\tabularnewline
\bottomrule
\end{longtable}

\begin{enumerate}
\def\labelenumi{\alph{enumi}.}
\tightlist
\item
  Which coefficient estimates are statistically significant? What are
  the t-statistics and p-values for each?
\item
  How would you interpret these results?
\item
  Critique the regression model. How would you improve it?
\end{enumerate}

\paragraph{Problem 2. 20pts. !!!}\label{problem-2.-20pts.}

Peter Gordon, in his talk at CCNY, presented results from linear
regressions to explain the growth of metropolitan areas. He begins with
a simple model to explain population growth from 1990-2000: Log
Population Growth 1990-2000 Coefficient t-stat p-value Constant term
-0.0229 -0.12 Population in 1990 (log) 0.0192 1.33 Pop. Density in 1990
-0.0504 -1.65 \% in manufacturing -0.0028 -1.63 R2 0.57 Where he also
includes dummy variables for Census Regions (New England, Mid Atlantic,
etc.). There are 79 observations and 67 degrees of freedom. a. What are
the p-values for the 3 coefficients? Are they significant? The averages
and standard deviations are: Average Standard deviation Population in
1990 (log) 14.52 14.89 Pop. Density in 1990 1.80 1.02 \% in
manufacturing 18.69 7.75 b. What is the predicted population growth for
a metropolitan area that is exactly average? c. What is the predicted
population growth for a metro area that is one standard deviation above
average in 1990 population? For a metro area one standard deviation
above average in density? In manufacturing concentration? d. Give a
careful explanation for why we would observe coefficients of these
signs.

\paragraph{Problem !!!}\label{problem-6}

Fill in the blanks in the following table showing SPSS regression
output. The model has the dependent variable as time spent working at
main job. Coefficients(a) Model Unstandardized Coefficients B Std. Error
t Sig. 1 (Constant) 198.987 7.556 26.336 .000 Female -65.559 4.031
\textbf{\emph{?}} \textbf{\emph{?}} African-American -9.190 6.190
\textbf{\emph{?}} \textbf{\emph{?}} Hispanic 17.283 6.387
\textbf{\emph{?}} \textbf{\emph{?}} Asian 1.157 12.137 \textbf{\emph{?}}
\textbf{\emph{?}} Native American/Alaskan Native -28.354 14.018 -2.023
.043 Education: High School Diploma \textbf{\emph{?}} 6.296 11.706 .000
Education: Some College \textbf{\emph{?}} 6.308 14.651 .000 Education:
4-year College Degree 110.064 \textbf{\emph{?}} 16.015 .000 Education:
Advanced degree 126.543 \textbf{\emph{?}} 15.714 .000 Age -1.907
\textbf{\emph{?}} -16.428 .000 a Dependent Variable: Time Working at
main job

\paragraph{Problem !!!}\label{problem-7}

You want to examine the impact of higher crude oil prices on American
driving habits during the past oil price spike. A regression of US
gasoline purchases on the price of crude oil as well as oil futures
gives the coefficients below. Critique the regression and explain
whether the necessary basic assumptions hold. Interpret each
coefficient; explain its meaning and significance. Coefficients(a) Model
Unstandardized Coefficients Standardized Coefficients B Std. Error Beta
t Sig. 1 (Constant) .252 .167 1.507 .134 return on crude futures, 1
month ahead .961 .099 .961 9.706 .000 return on crude futures, 2 months
ahead -.172 .369 -.159 -.466 .642 return on crude futures, 3 months
ahead .578 .668 .509 .864 .389 return on crude futures, 4 months ahead
-.397 .403 -.333 -.986 .326 US gasoline consumption -.178 .117 -.036
-1.515 .132 Spot Price Crude Oil Cushing, OK WTI FOB (Dollars per
Barrel) 4.23E-005 .000 .042 1.771 .079 a Dependent Variable: return on
crude spot price

\paragraph{Problem !!!}\label{problem-8}

Suppose I were to start a hedge fund, called KevinNeedsMoney Limited
Ventures, and I want to present evidence about how my fund did in the
past. I have data on my fund's returns, Rett, at each time period t, and
the returns on the market, Mktt. The graph below shows the relationship
of these two variables: a. I run a univariate OLS regression, Ret Mkt u
t t t = + +   0 1 . Approximately what value would be estimated for
the intercept term, 0? For the slope term, 1? 0.05 0.1 0.15 0.2 -0.15
-0.1 -0.05 0 0.05 0.1 0.15 Fund Return (Ret) Market Return (Mkt) b. How
would you describe this fund's performance, in non-technical language --
for instance if you were advising a retail investor without much finance
background?

\paragraph{Problem !!!}\label{problem-9}

Using the American Time Use Study (ATUS) we measure the amount of time
that each person reported that they slept. We run a regression to
attempt to determine the important factors, particularly to understand
whether richer people sleep more (is sleep a normal or inferior good)
and how sleep is affected by labor force participation. The SPSS output
is below. Coefficients(a) Model Unstandardized Coefficients B Std. Error
t Sig. 1 (Constant) -4.0717 4.6121 -0.883 0.377 female 23.6886 1.1551
20.508 0.000 African-American -8.5701 1.7136 -5.001 0.000 Hispanic
10.1015 1.7763 5.687 0.000 Asian -1.9768 3.3509 -0.590 0.555 Native
American/Alaskan Native -3.5777 3.8695 -0.925 0.355 Education: High
School Diploma 2.5587 1.8529 1.381 0.167 Education: Some College -0.3234
1.8760 -0.172 0.863 Education: 4-year College Degree -1.3564 2.0997
-0.646 0.518 Education: Advanced degree -3.3303 2.4595 -1.354 0.176
Weekly Earnings 0.000003 0.000012 -0.246 0.806 Number of children under
18 2.0776 0.5317 3.907 0.000 person is in the labor force -11.6706
1.7120 -6.817 0.000 has multiple jobs 0.4750 2.2325 0.213 0.832 works
part time 4.2267 1.8135 2.331 0.020 in school -5.4641 2.2993 -2.376
0.017 Age 1.1549 0.1974 5.850 0.000 Age-squared -0.0123 0.0020 -6.181
0.000 a. Which variables are statistically significant at the 5\% level?
At the 1\% level? b. Are there other variables that you think are
important and should be included in the regression? What are they, and
why?

\paragraph{Problem !!!}\label{problem-10}

\paragraph{Problem !!!}\label{problem-13}

A paper by Farber examined the choices of how many hours a taxidriver
would work, depending on a number of variables. His output is: ``Driver
Effects'' are fixed effects for the 21 different drivers. a. What is the
estimated elasticity of hours with respect to the wage? b. Is there a
significant change in hours on rainy days? On snowy days?

\paragraph{Problem !!!}\label{problem-14}

A study by Mehran and Tracy examined the relationship between stock
option grants and measures of the company's performance. They estimated
the following specification: Options = 0+1(Return on
Assets)+2(Employment)+3(Assets)+4(Loss)+u where the variable (Loss)
is a dummy variable for whether the firm had negative profits. They
estimated the following coefficients: Coefficient Standard Error Return
on Assets -34.4 4.7 Employment 3.3 15.5 Assets 343.1 221.8 Loss Dummy
24.2 5.0 Which estimate has the highest t-statistic (in absolute value)?
Which has the lowest p-value? Show your calculations. How would you
explain the estimate on the ``Loss'' dummy variable?

\paragraph{Problem !!!}\label{problem-15}

(20 points) Below is some SPSS output from a regression from the ATUS.
The data encompass only the group of people who report that they spent
non-zero time in education-related activities such as going to class or
doing homework for class. The regression examines the degree to which
education-time crowds out TV-watching time. The dependent is time spent
watching TV. The independents are time spent on all Education-related
activities as well as the usual demographic variables. Fill in the
blanks. Coefficients(a) Model Unstandardized Coefficients B Std. Error t
Sig. 1 (Constant) 160.531 14.658 10.952 .000 time spent on
Educationrelated activities -.137 .023 \textbf{?} \textbf{?} female
-26.604 7.852 \textbf{?} \textbf{?} African-American -4.498 \textbf{?}
-.417 .677 Hispanic \textbf{?} 12.181 -.681 .496 Asian -7.881 19.291
\textbf{?} \textbf{?} Native American/Alaskan Native -4.335 28.633 -.151
\textbf{?} Education: High School Diploma 1.461 13.415 .109 \textbf{?}
Education: Some College 3.186 \textbf{?} .311 .756 Education: 4-year
College Degree -47.769 13.471 -3.546 \textbf{?} Education: Advanced
degree \textbf{?} 18.212 -3.379 .001 Age \textbf{?} .276 2.839 .005
Weekly earnings {[}2 implied decimals{]} .000 .000 -.990 .322 In the
Labor Force -25.210 10.794 \textbf{?} .020 Has multiple jobs .918 15.299
\textbf{?} .952 Works part time 3.816 10.427 .366 .714 a Dependent
Variable: watching TV (not religious)

Using the same SPSS output from the regression above, explain clearly
which variables are statistically significant. Provide an interpretation
for each of the observed signs. What about the magnitude of the
coefficients? What additional variables (that are in the dataset) should
be included? What results are surprising to you? (Note your answer
should be a well-written few paragraphs, not just terse answers to the
above questions.)

\paragraph{Problem !!!}\label{problem-16}

A colleague proposes the following fitted line. Explain how or if his
model could be an OLS regression. There are 100 observations of pairs of
( x y i i i , , 1 100 ) = and for simplicity assume 0, 0 i i x y   for
all i =1 100 . For the first 99 observations, the fitted value, ˆ i y ,
is equal to the actual value, so ˆ ,1 1 99 i i y y = = . But for the
100th observation the fitted value misses the true value by 2, so 100
100 y y − = ˆ 2 . If the fitted values do not come from an OLS
regression, how should the fitted model be changed?

\subsection{5. Statistics using Datasets (R
required)}\label{statistics-using-datasets-r-required}

CI: Confidence Interval HT: Hypothesis tests kNN: k Nearest Neighbors

\paragraph{Problem 1: Taxi data. 30pts using OLS, kNN, HT,
CI}\label{problem-1-taxi-data.-30pts-using-ols-knn-ht-ci}

Using a subsample of the taxi data, I find that on weekends there were
193750 rides paid with credit cards and 187694 rides paid with cash.

\begin{enumerate}
\def\labelenumi{\alph{enumi}.}
\tightlist
\item
  Find a 90\% confidence interval for the fraction of rides paid in
  cash.
\item
  On weekdays there were 582335 rides paid with a card and 509798 paid
  in cash. What is a 90\% confidence interval for the fraction paid in
  cash now?
\item
  Are these proportions statistically significantly different? Explain
  and calculate t-stat and p-value.
\item
  What are some possible explanations? What data would you want to
  consider additionally? I'm not (yet) asking for data just an
  explanation of your thought process.
\item
  Using that data (the smaller sample is on Blackboard) can you
  construct a knn estimate of which fares are likely to tip more than
  15\%? OLS estimates of tip amount?
\end{enumerate}

\paragraph{Problem 2 w/ ATUS. 30pts using CI, OLS,
kNN}\label{problem-2-w-atus.-30pts-using-ci-ols-knn}

ATUS records the numbers of minutes in a typical day that people spend
on various activities. ACT\_WORK is the number of minutes spent working;
I'll define more than 420 minutes (7 hours) as fulltime.

\begin{longtable}[]{@{}lll@{}}
\toprule
number & Fulltime & Parttime\tabularnewline
\midrule
\endhead
Educ HS diploma & 8219 & 11879\tabularnewline
Educ some college & 6412 & 9989\tabularnewline
Educ Bachelor & 8065 & 12521\tabularnewline
\bottomrule
\end{longtable}

\begin{enumerate}
\def\labelenumi{\alph{enumi}.}
\tightlist
\item
  Conditional on the individual having a HS diploma, what fraction are
  working fulltime? What is a 95\% Confidence Interval?
\item
  Conditional on the individual working part-time, what fraction have a
  Bachelor's degree? What is the 95\% Confidence Interval?
\item
  Using the data on Blackboard, which is a subset of the ATUS data that
  selects people who are employed, can you find additional important
  factors in explaining the time spent at work? Explain with some OLS
  models and/or knn.
\end{enumerate}

\paragraph{Problem 3 w/ CEX. 30pts using OLS,
tables}\label{problem-3-w-cex.-30pts-using-ols-tables}

Consider the CEX data; estimate some models to explain APPARPQ,
expenditure on apparel in the previous quarter (includes MENBOYPQ,
WOMGRLPQ, and FOOTWRPQ - expenditure on apparel for Men and boys; Women
and girl; footwear). How important iseducational attainment on this
expenditure category?

\begin{enumerate}
\def\labelenumi{\alph{enumi}.}
\tightlist
\item
  What are conditional mean expenditure on apparel for different
  educational levels? What about conditional means for those who spent a
  non-zero amount?
\item
  Can you estimate some interesting OLS models and discuss the important
  variables in explaining apparel expenditure? Explain.
\item
  Can you estimate further useful models? Explain
\end{enumerate}

\paragraph{Problem 4.}\label{problem-4.-1}

(20 points) Consider the PUMS data for people in NY, that we've been
using in class. For now restrict attention to just working people
(explain how you might define that).

\begin{enumerate}
\def\labelenumi{\alph{enumi}.}
\tightlist
\item
  Do a statistical test of the difference in average age between working
  people in the Bronx vs working people in Brooklyn. What is the 95\%
  confidence interval for the difference in means?
\item
  What if you were using the Age data but regularized so that the min is
  zero and max is one {[}recall my function, (X\_in-min(X\_in,na.rm =
  TRUE))/abs(max(X\_in,na.rm = TRUE)-min(X\_in, na.rm =TRUE)){]}. Would
  the statistical test come out the same? Why or why not?
\end{enumerate}

\paragraph{Problem 5. 25pts using proportions,
tables}\label{problem-5.-25pts-using-proportions-tables}

I used the PUMS data to look at wages and commute type, getting this
table for people in the City: (you can answer parts a-c without R)

\begin{longtable}[]{@{}llll@{}}
\toprule
& bus & Car & Subway\tabularnewline
\midrule
\endhead
Wage below \$25,000 & 1501 & 2394 & 3704\tabularnewline
Wage above \$75,000 & 385 & 1825 & 2194\tabularnewline
\bottomrule
\end{longtable}

\begin{enumerate}
\def\labelenumi{\alph{enumi}.}
\tightlist
\item
  Given that someone takes the bus to work, what is the probability that
  they're making wages above \$75,000?
\item
  Given that someone takes the subway to work, what is the probability
  that they make wages below \$25,000?
\item
  Given that someone has wage above \$75,000, what is the probability
  that they drive a car to work?
\item
  Using the PUMS data, can you narrow this further - what are the
  socioeconomics of bus/subway in the various boroughs? What is the
  wealthiest PUMA area and how do the people living there tend to
  commute?
\end{enumerate}

\paragraph{Problem 6. 25pts using plots, tables, OLS
kNN}\label{problem-6.-25pts-using-plots-tables-ols-knn}

Use the CEX data that I provided and consider the fraction spent on
entertainment, ENTERTPQ/TOTEXPPQ.

\begin{enumerate}
\def\labelenumi{\alph{enumi}.}
\tightlist
\item
  Find some descriptive statistics about this fraction, for some
  subgroups. Tell me something interesting about this data. Are there
  sub-categories that explain some of the variation?
\item
  Create a histogram and/or density plot. What do these reveal?
\item
  Estimate a linear regression and discuss what this shows.
\item
  Estimate a k-nn classification to predict which households are in the
  lowest 25\% in terms of entertainment spending. Discuss what variables
  are important in classifying.
\end{enumerate}

\paragraph{Problem 7 w/ ATUS. 25pts using kNN, OLS,
HT}\label{problem-7-w-atus.-25pts-using-knn-ols-ht}

Using the ATUS data, describe the time spent working (ACT\_WORK). Do
people with more education work more or less hours than people with less
education? What other factors are important? You should choose a variety
of methods (perhaps including comparison of means, linear regression,
nearest neighbor) that demonstrate your econometric virtuosity.
Carefully specify the statistical tests that you perform, including the
null hypothesis and test statistics including t-stat and p-value.

\paragraph{Problem 8. !!!}\label{problem-8.}

\begin{enumerate}
\def\labelenumi{\arabic{enumi}.}
\setcounter{enumi}{13}
\tightlist
\item
  (20 points) A recent report asserted that people who worked more hours
  also tended to be fatter (among those in certain occupations). (The
  paper doesn't give precise numbers so I'll make them up -- don't
  bother with Google.) The paper did much more econometric analysis of
  course. Nevertheless, suppose that, of the 7219 women working
  non-strenuous occupations, 23\% are working more than 40 hours/week.
  Of those women in non-strenuous occupations working more than 40
  hours/week, 27.3\% were obese; of those women in non-strenuous
  occupations working less than 40 hours/week, 24.6\% were obese. There
  were also 714 women in strenuous occupations with 21\% working more
  than 40 hours/week. Of the women in strenuous occupations working more
  than 40 hours/week, 28.1\% were obese while 37.4\% were obese among
  those working fewer hours. Does it seem likely that overtime makes
  certain groups more likely to be obese? J Abramowitz, ``Working Hours,
  Body Mass Index, and Health Status: A Time Use Analysis''
\end{enumerate}

\paragraph{Problem 9. !!!}\label{problem-9.}

\begin{enumerate}
\def\labelenumi{\arabic{enumi}.}
\setcounter{enumi}{14}
\tightlist
\item
  (20 points) I used the CEX data to look at the fraction of spending
  going to health insurance. I get the following table, grouped by
  education of the reference person: \%Insurance No HS HS diploma Some
  college, no degree Assoc degree Bach degree Adv degree less than 10\%
  467 1385 1191 615 1181 521 11\% - 20\% 82 231 157 71 122 58 21\% -
  30\% 21 65 27 10 32 7 more than 30\% 8 18 14 1 3 2
\end{enumerate}

\begin{enumerate}
\def\labelenumi{\alph{enumi}.}
\tightlist
\item
  Conditional on the reference person having a college degree
  (Associate's, Bachelor's or Advanced), what fraction devote more than
  20\% of spending to health insurance?
\item
  Conditional on the reference person having less than a college degree,
  what fraction spend more than 20\% on health insurance?
\item
  Is this difference statistically significant?
\item
  What is the overall share (in this sample) of people with any college
  degree? What share of people spending more than 20\% is made up of
  people with any college degree?
\item
  Are those break points (+/- 20\%; any degree) reasonable? Can you
  suggest better? Explain.
\item
  What problems might there be, with the classification and analysis
  here? Can you do better with the CEX data?
\end{enumerate}

\paragraph{Problem 10. !!!}\label{problem-10.}

(25 points) After the Nobel Prize awards to Fama, Hansen, and Shiller,
we look at predictability of stock returns, using data on stocks in the
S\&P500. There are some days where many of these company's shares have
negative returns; other days where many have positive. In 2012, more
than 70\% of the companies had positive returns on about 25\% of the
days; on another 25\% of the days fewer than 30\% had ``up'' returns. On
the days following ``70\% up'' days, the average return was .06 percent,
with standard deviation of 1.72; on days following ``30\% up,'' the
average return was .10 percent, with standard deviation of 1.66. There
were 65 days of 70\% or more up; there were 59 days of 30\% or fewer up.
xxi. (1 pt) What is the difference in means? xxii. (2 pts) What is the
standard error of the difference in means? xxiii. (2 pts) Is this
difference statistically significant? What is the p-value? Explain.
xxiv. (20 pts) Using the data given on Blackboard, specify more
hypotheses about stock behavior and test these.

\paragraph{Problem 11. !!!}\label{problem-11.}

\begin{enumerate}
\def\labelenumi{\arabic{enumi}.}
\setcounter{enumi}{10}
\tightlist
\item
  (30 points) With the NSA spying revelations, we return to questions of
  whether there is wage discrimination
\end{enumerate}

against people with ancestry from the Middle East or North Africa
(MENA). I've created program in SPSS syntax and R that you can run,
which will define MENA\_ANC if the person's ancestry is from MENA
(except Israel) or MENA\_BPL if the person's birthplace is MENA. You
should consider whether there are differences in wages and incomes
between people from the MENA or others; of course one decision to make
is who is a relevant comparison group. Calculate averages between
groups, considering also things like education; which are statistically
significant? Explain in detail.

\paragraph{Problem !!!}\label{problem-25}

(20 points) Use the ATUS data (available from Blackboard) on the time
that people spend in different activities. o. Among households with
kids, what is the average time spent on activities related to kids?
p.~Among households with kids, how much time to men and women spend on
activities related to kids? Form a hypothesis test for whether there is
a statistically significant difference between the time that men and
women spend with kids. What is the p-value for the hypothesis of no
difference? What is a 95\% confidence interval for the difference in
time? q. Why do you think that we would find these results? Explain
(perhaps with some further empirical results from the same data set).

\paragraph{Problem !!!}\label{problem-26}

\begin{enumerate}
\def\labelenumi{\arabic{enumi}.}
\setcounter{enumi}{63}
\tightlist
\item
  Using the ATUS dataset that we've been using in class, form a
  comparison of the mean amount of TV time watched by two groups of
  people (you can define your own groups, based on any of race,
  ethnicity, gender, age, education, income, or other of your choice).
\end{enumerate}

\begin{enumerate}
\def\labelenumi{\alph{enumi}.}
\tightlist
\item
  What are the means for each group? What is the average difference?
\item
  What is the standard deviation of each mean? What is the standard
  error of each mean?
\item
  What is a 95\% confidence interval for each mean?
\item
  Is the difference statistically significant?
\end{enumerate}

\paragraph{Problem !!!}\label{problem-27}

Use the CPS dataset (available from Blackboard) to do a regression.
Explain why your dependent variable might be caused by your independent
variable(s). What additional variables (that are in the dataset) might
be included? Why did you exclude those? Next examine the regression
coefficients. Which ones are significant? Do the signs match what would
be predicted by theory? Are the magnitudes reasonable? (Note your answer
should be a well-written few paragraphs, not just terse answers to the
above questions. No SPSS output dumps either!)

\subsection{6. Regression Analysis using Datasets (R
Required)}\label{regression-analysis-using-datasets-r-required}

\paragraph{Problem 1 !!!}\label{problem-1-2}

(20 points) Use the Fed SCF 2010 data (available from Blackboard). This
is the Survey of Consumer Finances, which is not representative (without
using the weights, which you need not do for now) -- it intentionally
oversamples rich people to find out about their finances. Concentrate
for now on the variable ``SAVING'' (about the 100th variable in the
list) which is the amount that people have in their savings accounts. f.
Test the null hypothesis that there is no difference between people who
are older or younger than 65. What is the p-value for this test?

\paragraph{Problem 2!!!}\label{problem-2-1}

(25 points) Use the ATUS data (available from Blackboard) on the time
that people spend in different activities. Construct a linear regression
explaining the time that people spend on enjoyable activities (t\_enjoy
which includes most of the T12 items). Restrict the data to include only
those people spending a non-zero amount of time on such activities. g.
What are likely to be some of the most important determinants of time
spent on enjoyable activities? Which of these are in the ATUS data?
Should the person's wage be included (do you think income or
substitution effect would dominate)? What are some important
determinants, that you could imagine a survey measuring, that are not in
the ATUS data? You might find descriptive statistics for the included
variables. h. Carefully specify and estimate a linear regression. What
are the statistically significant coefficients? Which explanatory
variables are most important? Are there surprises? Discuss your results.
(You might want to estimate more models or create additional variables.)

\paragraph{Problem !!!}\label{problem-28}

(25 points) Use the PUMS data (available from Blackboard) on the
residents of NYC. Consider the time (in minutes) spent by people to
travel to work; this variable has name JWMNP. r. How many men and women
answered this question? What variables do you think would be relevant,
in trying to explain the variation in commuting times? s. Form a linear
regression with the dependent variable, ``JWMNP Travel Time to Work,''
and relevant independent variables. t. Which independent variables have
coefficients that are statistically significantly different from zero?

\paragraph{Problem !!!}\label{problem-29}

Use the SPSS dataset, atus\_tv from Blackboard, which is a subset of the
American Time Use survey. This time we want to find out which factors
are important in explaining whether people spend time watching TV. There
are a wide number of possible factors that influence this choice. a.
What fraction of the sample spend any time watching TV? Can you find
sub-groups that are significantly different? b. Estimate a regression
model that incorporates the important factors that influence TV viewing.
Incorporate at least one nonlinear or interaction term. Show the SPSS
output. Explain which variables are significant (if any). Give a short
explanation of the important results.

\paragraph{Problem !!!}\label{problem-30}

\begin{enumerate}
\def\labelenumi{\arabic{enumi}.}
\setcounter{enumi}{40}
\tightlist
\item
  Estimate the following regression:: S\&P100 returns = 0 + 1(lag
  S\&P100 returns) + 2(lag interest rates) + ε using the dataset,
  financials.sav. Explain which coefficients (if any) are significant
  and interpret them.
\end{enumerate}

\subsection{}\label{section}


\end{document}
